\chapter{Evaluation\label{cha:chapter6}}


\section{Test Environment\label{sec:eval_te_en}}
Normally in case of a concrete application there is no need to show in details how to install and configure the application but the output of this thesis is a framework which consist of many components. In order to make it easier for developers who need to deploy this framework, this section provide a details information on how to install and configure each component. Each component will be installed on a Linux machine running in the \ac{AWS} platform. Each of these machines has 7 GB RAM and 20 \ac{EC2} Compute Units (8 virtual cores with 2.5 \ac{EC2} Compute Units each) and running Ubuntu 12.04. The name of this machine within \ac{AWS} platform is \textit{High-CPU Extra Large Instance/c1.xlarge}. In order to have these machines in the same private network, these machines are created within the Amazon \ac{VPC}. 

Amazon \ac{VPC} enables provisioning of a logically isolated section of the AWS Cloud where one can launch AWS resources in a virtual network. One have complete control over the virtual networking environment, including selection of the IP address range, creation of subnets, and configuration of route tables.

Table \ref{tbl:ap_addresses} shows the IP address of each component. Thereby only the components, the app management and the content distribution have a private as well as a public IP address because these components as seen in figure \ref{fig:arch_overview} are the only components that the developer can interact with. 

\begin{table}[htb]
\begin{tabular}{|c|c|c|}
\hline 
Name & IP address \\ 
\hline 
App Management & $\begin{array}{l} \textbf{public:107.23.121.185} \\ \textbf{private:10.0.0.130}  \end{array}$ \\ 
\hline 
Repository & \textbf{10.0.0.135} \\ 
\hline 
Search Engine & \textbf{10.0.0.136} \\ 
\hline
Message Broker & \textbf{10.0.0.137} \\ 
\hline
Content Adaptation & \textbf{10.0.0.138} \\ 
\hline
Content Distribution & $\begin{array}{l} \textbf{public:107.23.180.173} \\ \textbf{private:10.0.0.139}  \end{array}$ \\ 
\hline
\end{tabular} 
\caption{IP address of each component}
\label{tbl:ap_addresses}
\end{table} 

The source code along with the documentation of the entire framework can be obtained from the following GitHub link: \url{https://github.com/abdul7383/cccd} .

\subsection{App Management\label{sec:eval_te_en_app}}
The source could of the app management component can be obtained with following command:

In order to deploy the app management component, one needs first to install and configure the following softwares:

\begin{itemize}
\item{\ac{JDK} 7:} Installing  \ac{JDK} 7 on Ubuntu is just easy as typing the command in listing \ref{lst:jdk_installation}.
\begin{code}
\begin{minted}[frame=single,tabsize=2,fontsize=\footnotesize]{console}
sudo apt-get install openjdk-7-jdk
\end{minted}
\caption{Installing \ac{JDK} 7 on Ubuntu}
\label{lst:jdk_installation}
\end{code}
\end{itemize}
bla bla

\subsection{App Managment\label{sec:eval_te_en_mongo}}
%The installation of MongoDB on Ubuntu is described in listing \ref{lst:mongdb_installation}, see this reference for more details \cite{MongoDb:installation}.
%\begin{code}
%\begin{minted}[frame=single,tabsize=2,fontsize=\footnotesize]{console}
%sudo apt-key adv --keyserver keyserver.ubuntu.com --recv 7F0CEB10
%sudo echo "deb http://downloads-distro.mongodb.org/repo/ubuntu-upstart dist 10gen" >> /etc/apt/sources.list.d/10gen.list
%sudo apt-get update
%sudo apt-get install mongodb-10gen
%\end{minted}
%\caption{Installing MongoDB on Ubuntu}
%\label{lst:mongdb_installation}
%\end{code}
\section{Test Scenarios\label{sec:eval__te_sc}}
	\subsection{Usability\label{seq:eval_usab}}
	
	\subsection{Performance\label{seq:eval_perf}}