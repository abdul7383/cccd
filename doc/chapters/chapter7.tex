\chapter{Conclusion\label{cha:chapter7}}
The objective of this thesis is to develop a contextual content management framework that supports the correlation and synchronization of context information and multimedia content and also provides an efficient and scaleable discovery and distribution mechanism. The framework facilitates the development of applications that need transcoding of contents, an easy scaled and distributed repository and also a powerful, ready for the cloud search engine in order to discover the data. 

In order to develop the desired framework, the general background of the different related technologies that were or might have been involved in this thesis was discussed in chapter \ref{cha:chapter2}. In the subsequent chapter \ref{cha:chapter3}, the requirements for the framework to be developed, were defined.

The framework prototype was implemented in chapter \ref{cha:chapter5} for the purpose of evaluating the design architecture presented in chapter \ref{cha:chapter4}. The design is characterized mainly by its high scalability. Thereby each component can be scaled independently, i.e. one can have two or more instances of the app management component that can be managed by a load balancer for distributing the incoming traffics across these instances, or one can have more instances of the repository, the search engine or the content distribution component for providing applications that are highly available and protected against single-instance failure.

Chapter \ref{cha:chapter6} provided two test scenarios, a usability and performance test, so that the implemented framework prototype could be evaluated.

One of the task of this thesis was to implement an application on a mobile device that evaluates the implemented framework, however, this task could not be done due to lack of time.
 
\section{Problems encountered}
The development of frameworks is not trivial. Compared to concrete conventional applications, frameworks are much more complex. The objective of developing a framework is to identify abstractions with a bottom-up approach, meaning that firstly one needs to provide some concrete applications, which will be used as use cases. These use cases can be used to illustrate better the functional assets of the framework and facilitate a deduction of its requirements.

Another problem faced in this thesis, was that the original approach was to use a ready open source content management system and try to build the framework upon it. This task was not at all easy, as most of these systems have a fixed-schema data model. In order to provide a generic framework that enables the user to easily update or extend the data model, there was no other option apart from starting to develop a new framework from ground level, based on schema-less databases. 

\section{Outlook}
It was not the purpose of this thesis to implement a market-ready product. Nevertheless, most of the components within the prototype implementation of this framework are based on technologies, i.e. Spring framework, MongoDB, Elasticsearch or Nginx, that are widely used in many production environments. The component which needs to be improved or even completely replaced is the content adaptation. This component was developed only to provide a proof of concept of a one use case scenario which needs a content transcoding service. The problem with this component is that it runs FFmpeg as a terminal command and it is not easy to determine if the command has been successfully processed. A better solution would be to use the C API of the FFmpeg platform for transcoding and segmenting videos.

Furthermore, the REST interfaces for the repository and the search engine should be extended to support their wide features.

In this thesis, Ontologies or RDF was not used intentionally so that the solution would not be unnecessarily more complex. However, in works that will be based on the results of this thesis in the future, it would be interesting to examine how the use of Ontologies or RDF could enhance the efficiency of the solution.