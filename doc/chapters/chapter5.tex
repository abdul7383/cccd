\chapter{Implementation\label{cha:chapter5}}
This chapter deals with the implementation of the design presented in Chapter 4. It was and is not the purpose of this work to implement a market-ready product. The focus is clearly on the conceptual part. To demonstrate the general realization of the proposed design, the implementation is done therefore only in the form of a prototype.


%\section{Development Environment\label{sec:impl_ecl}}	

\section{Tools \& Technologies\label{sec:impl_tools_tech}}	
This section gives an overview on the tools and technologies that have been used to simplify the implementation. The programming languages, which have been used to implement the framework, are Java and Python.

\subsection{Tools\label{sec:impl_tools}}
The following tools have been used to simplify, organize and test the implementation.

\paragraph{Eclipse Juno 4.2 IDE:\label{sec:impl_eclipse}} Eclipse is one of the widely used IDE for Java. It makes it easier to develop Java applications. 

\paragraph{Maven 3:\label{sec:impl_maven}} Maven is an open source build automation tool developed by the Apache Software Foundation. It uses an XML file called \textit{pom.xml} to describe the software project being built, its dependencies, the build order, directories, and required plug-ins. It comes with predefined targets for performing certain well-defined tasks such as compilation of code, its packaging and how and where to deploy the project.


\paragraph{Advanced REST client:\label{sec:impl_advanced_rest_cl}} Advanced REST client is a plugin within the Chrome browser and can help developers to create and test custom HTTP requests. It has been used mainly to test the different REST API, which have been developed within the framework.

\subsection{Technologies\label{sec:impl_technologies}}
In order to not reinvent the wheel again, most of the components in the framework are based on existing open source technologies. 

\paragraph{Spring Framework 3.1.1:\label{sec:impl_spring}} As described in \ref{sec:back_sp_fr}, the Spring framework is a lightweight solution and a very good base for building enterprise-ready applications. It provides an incredibly powerful and flexible collection of technologies and projects to improve enterprise Java application development. The following are some projects from the Spring framework, which have been used in developing the framework.

\subparagraph{Spring Security 3.1.1:\label{sec:impl_spring_sec}} Spring Security is a powerful and highly customizable authentication and access-control framework. It is the de-facto standard for securing Spring-based applications.

\subparagraph{Spring Data MongoDB 1.1.0:\label{sec:impl_spring_data}} Spring Data for MongoDB is part of the umbrella Spring Data project which aims to provide a familiar and consistent Spring-based programming model for for new datastores while retaining store-specific features and capabilities. The Spring Data MongoDB project provides integration with the MongoDB document database.

\subparagraph{Spring AMQP 1.1.3:\label{sec:impl_spring_amqp}} The Spring AMQP project applies core Spring concepts to the development of AMQP-based messaging solutions. It provide a "template" as a high-level abstraction for sending and receiving messages.

\paragraph{MongoDB 2.2.3:\label{sec:impl_mongo}} As described in \ref{sec:back_mongo}, MongoDB is a free and open-source document-oriented database and is completely schema-free and manages JSON-style documents.

\paragraph{Elasticsearch 0.20.4:} Elasticsearch is an open-source, distributed, RESTful, search engine built on top of Apache Lucene.Its data model roots lie with schema-free and document-oriented databases, and as shown by the NoSQL movement, this model proves very effective for building applications.

\paragraph{RabbitMQ 3.0.1:} RabbitMQ is an open-source message broker, which implements the AMQP standard. It provides  robust messaging services for applications and is reliable and highly scalable.

\paragraph{NginX 1.2.6:} Nginx- engine-x pronounced - is a free, open-source, high-performance HTTP server. Unlike traditional servers, Nginx does not rely on threads to handle requests. Instead it uses a much more scalable event-driven (asynchronous) architecture. This architecture uses small, but more importantly, predictable amounts of memory under load.

\paragraph{FFmpeg 0.9.2:} FFmpeg is the leading multimedia framework, able to decode, encode, transcode, mux, demux, stream, filter and play pretty much anything that humans and machines have created. It is an open source project licensed under LGPL version 2.1.
\pagebreak
\pagebreak

%\section{Used Open Source Tools\label{sec:impl_used_op_sr}}
\section{Framework Components\label{sec:impl_used_op_sr}}
This section describes how each component in the framework got implemented and realized.

\subsection{App Managment\label{sec:impl_app_man}}
The \textit{app management} component is developed completely in Java and it is based on the Spring framework. Figure bla shows the project structure of this component withing the Eclipse IDE.

As shown in figure bla, the project consist of three java packages, namely \textit{de.fhg.fokus.ngni.cccd.model}, 	\textit{de.fhg.fokus.ngni.cccd.rest} and \textit{de.fhg.fokus.ngni.cccd.services}, and also four configurations file, which are, the maven configuration file \textit{pom.xml}, the \textit{web.xml} file, the logging configuration file \textit{log4j.xml} and the Spring configuration file \textit{cccd-config.xml}.

The maven configuration file \textit{pom.xml} contains information about the project and configuration details used by Maven to build the project and also manages the dependencies which are needed for building the project.

Within the \textit{web.xml} file, one can set how to map a specific URL to a particular servlet and Spring provides this in the following form:

\begin{code}
\begin{minted}[frame=single,tabsize=2]{xml}
<web-app>
	<servlet>
		<servlet-name>cccd</servlet-name>
		<servlet-class>
			org.springframework.web.servlet.DispatcherServlet
		</servlet-class>
		<init-param>
			<param-name>
				contextConfigLocation
			</param-name>
			<param-value></param-value>
		</init-param>
		<load-on-startup>1</load-on-startup>
	</servlet>

	<servlet-mapping>
		<servlet-name>cccd</servlet-name>
		<url-pattern>/</url-pattern>
	</servlet-mapping>
</web-app>
\end{minted}
\end{code}

In the above example at the servlet-mapping tag, this says that all URL("/") being requests should be handled by the \textit{cccd} servlet.  And what is the \textit{cccd} servlet and where to find it is declared in the servlet tag which tells Tomcat which Java class it should resolve this \textit{cccd} servlet to. Normally in none Spring applications one could just directly specify a class which inherits from the HttpServlet class. In Spring, however, this is where it actually enter the Spring Framework.  Instead of defining the class, which needs to be executed for this servlet directly, one need to specify only the \textit{org.springframework.web.servlet.DispatcherServlet}. From this point onwards, the request and response are known by the Spring framework, so that one can apply Spring pre-processing and post-processing with special Spring modules, i.e. Security, Aspect Oriented Programming and so on.




	\subsection{Repository\label{sec:des_repo}}
	\subsubsection{MongoDB\label{sec:impl_mon}}
	
	\subsection{Search Engine\label{sec:des_se_en}}
	\subsubsection{ElasticSearch\label{sec:impl_el_se}}
	
	\subsection{Content Adaptation\label{sec:des_ar_ov}}	
	\subsubsection{FFmpeg\label{sec:impl_ff}}
	
	\subsection{Content Distribution\label{sec:des_cdn}}
	\subsubsection{HTTP-Live-Video-Stream-Segmenter-and-Distributor\label{sec:impl_http_li}}
	\subsubsection{NginX\label{sec:impl_ngi}}
	
	\subsection{Application Messaging\label{sec:des_me}}
	\subsubsection{RabbitMQ\label{sec:impl_ra_mq}}
	
	\subsection{User API\label{sec:des_api}}
	\subsubsection{SPRING DATA - REST\label{sec:des_api}}

\section{Components Integration and Configuration\label{sec:impl_comp_in}}

\section{REST API\label{sec:impl_rest_api}}