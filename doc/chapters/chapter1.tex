\chapter{Introduction\label{cha:chapter1}}

The Internet has become an essential part of our daily lives in different sectors business, social communications, healthcare, etc. It has revolutionized our economy and society and being therefore considered at the top of the technological revolution in the current century. The success of the Internet can be seen on its traffic growth. The monthly global Internet traffic is expected to quadruple between 2010 and 2015 growing up from about 20.2 exabytes to 80.5 exabytes (one exabyte equals one billion gigabytes)\cite{cisco1}. Indeed, this growth indicates how huge the content is (and being increased rapidly) that is uploaded and consumed. Around one million minutes of video content will cross the network per second in 2015 \cite{cisco1}. Around one hour You-tube videos are being uploaded per second and more than four billion views per day\cite{youtube1}. The video-sharing content in You-tube is only one example of a huge number of distributed contents on the Internet provided by various content delivery platforms. These platforms provide different types of contents, i.e. contents are heterogeneous and can be anything, e.g. video, multimedia, books, etc. 

The recent growth of multimedia content offered by multiple professional content providers (e.g. IPTV or mobile TV provider), available in several multimedia-based social networking communities or distributed in various user devices seems to be clear evidence for the need of an efficient multimedia provisioning framework that supports efficient and personalized provisioning and discovery mechanisms of multimedia content information comparing to the classical client-server provisioning systems. This thesis will address arise from the wealth of distributed multimedia content either in any controllable network or in user private network. The challenge is to provide users with technical means for rapid and instant access to relevant, trustworthy multimedia content information and enriched personalization.

Nowadays devices (e.g. PCs, smartphones, positioning devices, health monitors) in our environment are expected to work on high levels of independence, performing programmed actions that benefit their users in everyday life.In order to meet the set of requirements, these different
devices are connected together performing certain tasks. This concept is known as Machine-to-Machine (M2M) communication. M2M is a concept that defines the rules and relations between devices while cooperating. It implies a highly automated usage of a set of devices simultaneously, without much need for human interaction. Although with the increase of computational power, now it is even possible to run different M2M tasks on various consumer electronics (e.g. television sets, set-top boxes), smartphones are still more frequent used in M2M domain.

The aim of this thesis is to develop a context-aware platform in which context information of multimedia content environment (such as location and time) is considered and embedded into the captured multimedia content as content information or metadata. The proposed platform uses M2M concept, which provides several resources such as sensor informations like temperature, GPS data, and so on. These resources can be used to enrich the metadata of the captured content. The motivation behind this work is due to the lack of context/content aware storage management in the current Internet architecture which is one of its fundamental limitations \cite{ec1}. Consumer, who is interested in certain multimedia content service, submits the request with defined conditions (e.g. time, location, content provider, etc.) that are evaluated against multimedia context information in order to deliver the associated multimedia content information (e.g. content resources, description, etc.). User-specific conditions can be published once or updated regularly. The multimedia content service shall examine user-specific conditions and notifies the user with matched multimedia content. Therefore, an efficient interactions model between consumer and the service will be developed. Furthermore, an end-to-end multimedia content service will be implemented in order to demonstrate the developed concept.

\section{Motivation\label{sec:moti}}

Context-aware systems can help people in many areas of daily life to plan the daily schedule, to make important decisions correctly and perform other tasks instead of user.

Due to the fact that increased computing power of today's mobile devices, they perform tasks that were still a few years ago not possible. For these devices, with their increasingly complex applications, context-aware behaviour is of importance. Reliable and easy-used context-aware systems are required because of the explosive growth of content consumption from mobile and social interfaces and the consumer expectation of content availability. According to a statistic result reported by The Nielsen Company, U.S mobile video viewers have grown from 23 millions in the third quarter of 2010 to 31 millions in the third quarter of 2011.

M2M technology is growing fast and in the near future it will be available everywhere. Such a technology will help to enrich context information associated with multimedia content.

There are several reasons for working on this diploma thesis. On one hand, there is no correlation between contents and contexts nowadays supported by the current Internet architecture. However, there is a demand for solutions or products that simplify the usage of the distributed content based on its contexts. The unavailability of such solutions or products is one of the main motivations behind this thesis. Within the context of this thesis, a new valid prototyping for context-aware content management platform will be developed. Therefore this thesis can set the ground for future investigation and can further be used as a cornerstone and give directions for design of better and generally accepted.

On the other hand, working on this thesis gives me the chance to get in-depth knowledge and hands-on experience of a hot topic that will evolve, improve, develop in the years to come and eventually will become inevitable part of normal way of living.

\section{Objective\label{sec:objective}}

The main objective of this thesis is to develop a contextual content management platform in which the context information (metadata) is created automatically during content capturing and relying on local and distributed sensing information. The platform will support the correlation and synchronization of context information and multimedia content stream. To deliver the content to various devices the distribution mechanism will be implemented. The distribution mechanism has to be aware on device properties i.e screen resolution or internet speed. For context enrichment, the platform will use the benefits of M2M concept.

Figure \ref{fig:oarch} shows the main component of the platform:
\begin{figure}[htb]
  \centering
  \includegraphics[scale=0.5]{DA-OverallArchitecture.jpg}\\
  \caption{Overall Architecture}
  \label{fig:oarch}
\end{figure}


\section{Scope\label{sec:scope}}

Due to lack of time and the possible wide range of technologies which have been mentioned above, the objective will be defined here in details to decide what should be developed in this thesis.

The scope of this thesis is to manage the relationship between content and context. For managing this relationship the thesis will study the state of the art of data model description and then choose the appreciate one. The thesis will also integrate M2M concept for context enrichment.


The activities in this thesis are outlined as follows:
	\begin{itemize}
		\item Act. 1: Study the state of the art of data model format, context description language
				\vspace{-0.1in} 
		\item Act. 2: Defining concept for discovering sensors (locally deployed or with M2M platform), subscriptions and notifications for sensor information.
				\vspace{-0.1in} 
		\item Act. 3: Investigate the process of automatic creation of context information and data fusion (correlation between context and content)
				\vspace{-0.1in} 		
		\item Act. 4: Design the required management and delivery platforms 
    			\vspace{-0.1in} 
		\item Act. 5: Examine the available open sources for content management systems and HTTP-based streaming servers
 				\vspace{-0.1in} 
		\item Act. 6: Based on available open source solutions, develop an end-to-end platform that enables content provider(mobile app.) to capture multimedia content with the associated context information and publish this content to the server and allows consumers(mobile app. or web based) to discover and subscribe to multimedia content according to defined conditions. 
    			\vspace{-0.1in} 
    	\item Act. 7: Develop an end-to-end application to evaluate the implemented functionality
		\item Act. 8: Validate the implementation through an end-to-end deployment scenario that is planned to be deployed in the FOKUS open EPC(Evolved Packet Core) or MTC(Machine Type Communication) playground. In the first deployment scenario, the quality of content streaming to the consumer or network selection will be instantly identified according to user and network policies. In the second deployment scenario, content-related context information – in particular geographical and time information of multimedia content stream – can be considered as collaborative crowd sourcing with multimedia content that can enrich any MTC platform.
    \end{itemize}

\pagebreak

Figure \ref{fig:inout} shows inputs and outputs for this thesis:

\begin{figure}[htb]
  \centering
  \includegraphics[scale=0.4]{Inp_Out.jpg}\\
  \caption{Inputs \& Outputs}
  \label{fig:inout}
\end{figure}

By studying these tasks the goal is to draw conclusions about best practices in this domain, and design the platform that can establish the basis for its further development and future implementation.

%\section{Impact\label{sec:intro_impact}}

\section{Methodologie\label{sec:intro_meth}}
\section{Outline\label{sec:intro_out}}
%\section{Use case\label{sec:use_case}}
%
%For better understanding the functionality of the proposed platform, an example is discussed in this section as a use case scenario.
%
%A user as a content provider captures a live event (e.g. demonstration, car race, marathon, Tour de France, etc.) using an application on a GPS capable smartphone. While capturing, the application also collects context information (e.g. location, acceleration, temperature, time, etc.). Later the user uploads the captured content with its context information to the platform.
%
%Let's consider the Tour de France as an example for the uploaded content in the following. Any content consumer, who is interested in a specific uploaded content that has been captured in a specific place on the road of the Tour (e.g. Les Essarts town which is located in western France), searches for the content by specifying some related information such as time range, location and `Tour de France` as search string. The platform will then give the user a list of all content that match the specified criteria. The user can select any of the provided contents.
%
%\section{Requirement\label{sec:use_case}}
%The above use case is a textual representation illustrating a sequence of events, which imply different requirements. Figure \ref{fig:req} shows the requirement.
%\begin{figure}[htb]
%  \centering
%  \includegraphics[scale=0.4]{Requirements.jpg}\\
%  \caption{Requirements}
%  \label{fig:req}
%\end{figure}

%\section{Plan\label{sec:plan}}
%
%The processing time of a thesis from the application until the completion is six months. A rough schedule with milestones is shown in Figure \ref{fig:plan}. The chart provides a big picture for the conduct of the dissertation, which will be refined over time.
%During the processing time there are regular meetings between the candidate and supervisors.
%\begin{figure}[htb]
%  \centering
%  \includegraphics[scale=0.6]{Plan_new1.PNG}\\
%  \caption{Plan}
%  \label{fig:plan}
%\end{figure}
