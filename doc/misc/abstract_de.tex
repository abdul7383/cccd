\thispagestyle{empty}
\vspace*{0.2cm}

\begin{center}
    \textbf{Zusammenfassung}
\end{center}

\vspace*{0.2cm}

\noindent 
Kontextbasierte Systeme k\"onnen Menschen in vielen Bereichen des t\"aglichen Lebens unterst\"utzen, wie z.B. bei der Planung des Tagesablaufs, beim Treffen von wichtigen Entscheidungen oder beim Erledigen sonstiger Aufgaben anstelle des Benutzers. Die heutigen mobilen Ger\"ate f\"uhren aufgrund ihrer hohen Rechenleistungen Aufgaben aus, die noch vor einigen Jahren nicht m\"oglich waren. F\"ur diese Ger\"ate mit ihren immer komplexeren Anwendungen ist kontextsensitives Verhalten von Bedeutung. Das explosive Wachstum des Verbrauchs von digitalen Inhalten auf mobilen Geräten und sozialen Netzwerken und die Erwartungen der Verbraucher, dass die Inhalte auf verschiedenen Ger\"aten verf\"ugbar sein sollen, sollte als Beweis genommen oder zumindest als deutliches Zeichen angesehen werden, für die Notwendigkeit eines zuverl\"assigen, leichten, skalierbaren und kontextsensitiven Systems. Einem statistischen Bericht zur Folge, führte die wachsende Beliebtheit von Smartphones auch zu einem dramatischen Anstieg der mobilen Videos. W\"ahrend im Jahr 2010 23 Millionen US-Mobilfunkkunden sich Videos auf ihren Handys ansahen, waren es im Jahr 2011 31 Millionen Kunden, was einer Zunahme von 35 \% entspricht \cite{mobile-media-report}.

\ \
\ \
Im Rahmen dieser Diplomarbeit wird ein Framework entwickelt, dass die Entwicklung von kontextsensitiven Anwendungen erleichtert. Dabei ist das Framework nicht auf bestimmte Kontextinformationen oder Anwendungsdom\"anen beschr\"ankt, sondern eignet sich für viele verschiedene kontextsensitive Anwendungen.

Zun\"achst werden die Begriffe Kontext und Kontextsensivit\"at wie auch die relevanten Techniken eingef\"uhrt und erl\"autert. Anschlie{\ss}end werden die Anforderungen an ein solches Framework analysiert und definiert. Basierend auf diesen Anforderungen wird eine Konzeption für das Framework entwickelt und seine Machbarkeit dann erfolgreich durch eine prototypische Implementierung evaluiert.


%Kontextbasierte Systeme k\"onnen Menschen in vielen Bereichen des t\"aglichen Lebens helefen, wie z.B. bei der Planung des Tagesablauf, wichtige Entscheidungen richtig zu treffen oder sonstige Aufgaben anstelle des Benutzers zu erledigen.
%Aufgrund der Tatsache, dass die heutigen mobilen Ger\"aten \"uber h\"ohe Rechenleistung verf\"ugen, f\"uhren sie Aufgaben, die noch vor einigen Jahren nicht m\"oglich. F\"ur diese Ger\"ate mit ihren immer komplexeren Anwendungen ist kontextsensitive Verhalten von Bedeutung. Das explosive Wachstum des Verbrauchs von digitalen Inhalten von mobilen und sozialen Schnittstellen und die Erwartungen der Verbraucher, dass die Inhalten auf verschiedenen Ger\"aten verf\"ugbar sein sollen, sollte als Beweis genommen werden, oder zumindest als deutliches Zeichen, für die Notwendigkeit einem zuverl\"assigen, leichten und skalierbaren kontextsensitive System. Statische Bericht zur Folge, dass die wachsende Beliebtheit von Smartphones auch zu einem dramatischen Anstieg der mobilen Videos gef\"uhrt hat. W\"ahrend 23 Millionen US-Mobilfunkkunden Videos auf ihren Handys im Jahr 2010 angesehen haben, waren Sie im Jahr 2011 31 Millionen Kunden, was eine Zunahme von 35 \% entspricht \cite{mobile-media-report}.
%
%\ \
%\ \
%Im Rahmen dieser Diplomarbeit wird ein Framework entwickelt, dass die Entwicklung von kontextsensitiven Anwendungen erleichtert. Dabei ist das Framework nicht auf bestimmte Kontextinformationen oder Anwendungsdom\"anen beschr\"ankt, sondern eignet sich für viele verschiedene kontextsensitive Anwendungen.
%
%Zun\"achst werden die Begriffe Kontext und Kontextsensivit\"at wie auch die relevanten Techniken eingef\"uhrt. Im Folgenden werden die Anforderungen an ein solchen Framework analysiert und definiert. Basierend auf diesen Anforderungen wird eine Konzeption für das Framework entwickelt und seine Machbarkeit dann erfolgreich durch eine prototypische Implementierung evaluiert.