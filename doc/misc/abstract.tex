\thispagestyle{empty}
\vspace*{1.0cm}

\begin{center}
    \textbf{Abstract}
\end{center}

\vspace*{0.5cm}

\noindent
Context-aware systems can help people in many areas of daily life in order to plan their daily schedule, to make the right decisions  and to perform other tasks for them.
Due to the fact of the increased computing power of today's mobile devices, they perform tasks that were not possible a few years ago. For these devices, with their increasingly complex applications, context-aware behavior is of great importance. The explosive growth of content consumption from mobile devices and social networks and the consumer expectation of content availability on various devices should be taken as evidence, or at least as a strong indication, for the need of reliable, easy usable and scalable context-aware systems.%Reliable and easy-used context-aware systems are required because of the explosive growth of content consumption from mobile and social interfaces and the consumer expectation of content availability. 
According to a statistical result reported by The Nielsen Company, the growing popularity of smartphones has also led to a dramatic rise in mobile videos. While 23 million U.S mobile subscribers viewed videos on their phones in 2010, 31 million viewed mobile videos in 2011 - a 35\% increase \cite{mobile-media-report}. 
%U.S mobile video viewers have grown from 23 millions in the third quarter of 2010 to 31 millions in the third quarter of 2011.
\\
\\
In this thesis, a framework is developed that facilitates the development of context-aware applications. Thereby, the framework is not limited to specific context information or application domain and therefore suitable for many context-aware applications.

First of all, the terms context and context-awareness plus the relevant technologies are introduced. In the following, the requirements for such a framework are analyzed and defined. Based on these requirements, a conceptual design for the framework is developed and its feasibility is evaluated successfully by means of a prototypical implementation.

%On the one hand this PDF should give a guidance to people who will soon start to write their thesis. The overall structure is explained by examples. On the other hand this text is provided as a collection of LaTeX files that can be used as a template for a new thesis. Feel free to edit the design.
%\\
%\\
%It is highly recommended to write your thesis with LaTeX. I prefer to use Miktex in combination with TeXnicCenter (both freeware) but you can use any other LaTeX software as well. For managing the references I use the open-source tool jabref. For diagrams and graphs I tend to use MS Visio with PDF plugin. Images look much better when saved as vector images. For logos and 'external' images use JPG or PNG. In your thesis you should try to explain as much as possible with the help of images.
%\\
%\\
%The abstract is the most important part of your thesis. Take your time to write it as good as possible. Abstract should have no more than one page. It is normal to rewrite the abstract again and again, so  probaly you won't write the final abstract before the last week of due-date. Before submitting your thesis you should give at least the abstract, the introduction and the conclusion to a native english speaker. It is likely that almost no one will read your thesis as a whole but most people will read the abstract, the introduction and the conclusion.
%\\
%\\
%Start with some introductionary lines, followed by some words why your topic is relevant and why your solution is needed concluding with 'what I have done'. Don't use too many buzzwords. The abstract may also be read by people who are not familiar with your topic.