\thispagestyle{empty}
\vspace*{1.0cm}

\begin{center}
    \textbf{Abstract}
\end{center}

\vspace*{0.5cm}

%\noindent
Context-aware systems can help people in many areas of daily life in order to plan their daily schedule, to make the right decisions  and to perform other tasks for them.
Due to the fact of the increased computing power of today's mobile devices, they perform tasks that were not possible a few years ago. For these devices, with their increasingly complex applications, context-aware behavior is of great importance. The explosive growth of content consumption from mobile devices and social networks and the consumer expectation of content availability on various devices should be taken as evidence, or at least as a strong indication, for the need of an efficient multimedia provisioning framework that supports efficient and personalized provisioning and discovery mechanisms of multimedia content information. According to a statistical result reported by The Nielsen Company, the growing popularity of smartphones has also led to a dramatic rise in mobile videos. While 23 million U.S mobile subscribers viewed videos on their phones in 2010, 31 million viewed mobile videos in 2011 - a 35\% increase \cite{mobile-media-report}. 
\\
\\
In this thesis, a framework is developed that facilitates the development of context-aware applications. The framework supports the correlation of context information and multimedia content and also provides an efficient and scaleable discovery and distribution mechanism. Thereby, the framework is not limited to specific context information or application domain and therefore suitable for many context-aware applications.

First of all, the terms context and context-awareness plus the relevant technologies are introduced. In the following, the requirements for such a framework are analyzed and defined. Based on these requirements, a conceptual design for the framework is developed and its feasibility is evaluated successfully by means of a prototypical implementation.
